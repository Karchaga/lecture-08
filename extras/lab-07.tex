\documentclass{tufte-handout}

\usepackage{xcolor}

% set image attributes:
\usepackage{graphicx}
\graphicspath{ {images/} }

% set hyperlink attributes
\hypersetup{colorlinks}

% set table attributes
\usepackage{tabu}
\usepackage{booktabs}

% set list attributes:
\usepackage{enumerate}
\usepackage{enumitem}


% ========================================================

% define the title
\title{SOC 4650/5650: Lab-07 - Weather Data Organization}
\author{Christopher Prener, Ph.D.}
\date{Spring 2020}
% =======================================================
\begin{document}
% =======================================================
\maketitle % generates the title
% =======================================================

\section{Directions}
Using data from the \texttt{data/lab-07/} folder available in the \texttt{lecture-08} repository, create several maps using RStudio. Your entire project folder system should be uploaded to GitHub by Monday, March 9\textsuperscript{th} at 5:00pm.

\vspace{5mm}
\section{Analysis Development}
The goal of this section is to create a self contained project directory with all of the data, code, map documents, results, and documentation a project needs.

\begin{enumerate}[label=\alph*.]
\item \textbf{Clone} the \texttt{lecture-07} repository if you have not already done so.
\item Create a project folder system with all of the necessary components, and drag the lab data from \texttt{lecture-07/data/lab-07/} into your RStudio Project's \texttt{data/} subdirectory. Also add an ArcGIS Pro project to this same directory, and move the default geodatabase into your \texttt{data/} subdirectory.
\item Create a \texttt{README.md} text file. Add a quick description of your project and outline the key directories and files that are included. 
\item Create a well-formatted RMarkdown document for your data cleaning efforts.
\item in RStudio, load the three pieces of raw data - \texttt{MO\_BOUNDARY\_Counties.shp}, \texttt{MO\_DEMOS\_CountyPop.shp}, and \texttt{MO\_DEMOS\_CountyDisability.csv}. 
\end{enumerate}

\vspace{5mm}
\section{Part 1: Data Wrangling}
\begin{enumerate}
\item Begin by creating a pipeline in RStudio that takes the county disability data and:
\begin{enumerate}
\item Creates a new variable that adds the total values for \texttt{under50\_dis}, \texttt{btwn50\_99\_dis}, and \texttt{btwn100\_150\_dis} - this is the total number of people living with the county who have a disability and live between 0\% and 150\% of the poverty line.
\item Removes all columns except the \texttt{GEOID} and the new variable you calculated.
\end{enumerate}
\item Next, remove the \texttt{NAME} column from the county population data and convert it from a \texttt{sf} object to a regular data frame.
\item Then, join the county population and disability data together.
\item After the data are joined, create a new variable that contains the the number of people per county who have a disability and live between 0\% and 150\% of the poverty line divided by the total number of people living in the county - the ratio of people we're particularly concerned about in in-climate weather to the total population of the county.
\item At this point, there should be four columns in the data set in this order - \texttt{GEOID}, the new variable you created with the county of people in poverty who have a disability, the total population, and the ratio of those two variables. Re-order your columns if necessary.
\item Remove all columns from the county shapefile data except the unique identification variable that matches the unique identification variable in the demographic data set you've created.
\item Join the county geometry to the demographic data set.
\item Calculate the square kilometers per county by getting the area of county from the \texttt{geometry} column and converting it from square meters to square kilometers.
\item Write the data as a shapefile to your \texttt{data/} folder.
\end{enumerate}

\vspace{5mm}
\section{Part 2: Geodatabase Creation}
Add your new shapefile and as well as the metro weather event shapefiles for tornado and hail events to your ArcGIS Project's geodatabase in the \texttt{data/} subdirectory.

\vspace{5mm}
\section{Analysis Development Follow-up}
Don't forget to knit your document when you are done! Also be sure to go back and update your \texttt{README.md} file with any changes to your project's organization or contents.


% =======================================================
\end{document}